\documentclass[floatsintext,man]{apa6}

\usepackage{amssymb,amsmath}
\usepackage{ifxetex,ifluatex}
\usepackage{fixltx2e} % provides \textsubscript
\ifnum 0\ifxetex 1\fi\ifluatex 1\fi=0 % if pdftex
  \usepackage[T1]{fontenc}
  \usepackage[utf8]{inputenc}
\else % if luatex or xelatex
  \ifxetex
    \usepackage{mathspec}
    \usepackage{xltxtra,xunicode}
  \else
    \usepackage{fontspec}
  \fi
  \defaultfontfeatures{Mapping=tex-text,Scale=MatchLowercase}
  \newcommand{\euro}{€}
\fi
% use upquote if available, for straight quotes in verbatim environments
\IfFileExists{upquote.sty}{\usepackage{upquote}}{}
% use microtype if available
\IfFileExists{microtype.sty}{\usepackage{microtype}}{}

% Table formatting
\usepackage{longtable, booktabs}
\usepackage{lscape}
% \usepackage[counterclockwise]{rotating}   % Landscape page setup for large tables
\usepackage{multirow}		% Table styling
\usepackage{tabularx}		% Control Column width
\usepackage[flushleft]{threeparttable}	% Allows for three part tables with a specified notes section
\usepackage{threeparttablex}            % Lets threeparttable work with longtable

% Create new environments so endfloat can handle them
% \newenvironment{ltable}
%   {\begin{landscape}\begin{center}\begin{threeparttable}}
%   {\end{threeparttable}\end{center}\end{landscape}}

\newenvironment{lltable}
  {\begin{landscape}\begin{center}\begin{ThreePartTable}}
  {\end{ThreePartTable}\end{center}\end{landscape}}




% The following enables adjusting longtable caption width to table width
% Solution found at http://golatex.de/longtable-mit-caption-so-breit-wie-die-tabelle-t15767.html
\makeatletter
\newcommand\LastLTentrywidth{1em}
\newlength\longtablewidth
\setlength{\longtablewidth}{1in}
\newcommand\getlongtablewidth{%
 \begingroup
  \ifcsname LT@\roman{LT@tables}\endcsname
  \global\longtablewidth=0pt
  \renewcommand\LT@entry[2]{\global\advance\longtablewidth by ##2\relax\gdef\LastLTentrywidth{##2}}%
  \@nameuse{LT@\roman{LT@tables}}%
  \fi
\endgroup}


\ifxetex
  \usepackage[setpagesize=false, % page size defined by xetex
              unicode=false, % unicode breaks when used with xetex
              xetex]{hyperref}
\else
  \usepackage[unicode=true]{hyperref}
\fi
\hypersetup{breaklinks=true,
            pdfauthor={},
            pdftitle={The Development of Color Terms in Shipibo-Konibo Children},
            colorlinks=true,
            citecolor=blue,
            urlcolor=blue,
            linkcolor=black,
            pdfborder={0 0 0}}
\urlstyle{same}  % don't use monospace font for urls

\setlength{\parindent}{0pt}
%\setlength{\parskip}{0pt plus 0pt minus 0pt}

\setlength{\emergencystretch}{3em}  % prevent overfull lines


% Manuscript styling
\captionsetup{font=singlespacing,justification=justified}
\usepackage{csquotes}
\usepackage{upgreek}

 % Line numbering
  \usepackage{lineno}
  \linenumbers


\usepackage{tikz} % Variable definition to generate author note

% fix for \tightlist problem in pandoc 1.14
\providecommand{\tightlist}{%
  \setlength{\itemsep}{0pt}\setlength{\parskip}{0pt}}

% Essential manuscript parts
  \title{The Development of Color Terms in Shipibo-Konibo Children}

  \shorttitle{Color Terms in Shipibo-Konibo Children}


  \author{Danielle Kellier*\textsuperscript{1}, Martin Fortier*\textsuperscript{2}, Maria Fernández Flecha\textsuperscript{3}, \& Michael C. Frank\textsuperscript{4}}

  % \def\affdep{{"", "", "", ""}}%
  % \def\affcity{{"", "", "", ""}}%

  \affiliation{
    \vspace{0.5cm}
          \textsuperscript{1} University of Pennsylvania\\
          \textsuperscript{2} PSL Research University\\
          \textsuperscript{2} Pontificia Universidad Católica del Perú\\
          \textsuperscript{2} Stanford University  }

  \authornote{
    \begin{itemize}
    \tightlist
    \item
      these authors contributed equally.
    \end{itemize}
    
    MCF was supported by a Jacobs Foundation Fellowship. We are especially
    grateful to Paul Kay for assistance throughout the study.
    
    Correspondence concerning this article should be addressed to Martin
    Fortier*, Postal address. E-mail:
    \href{mailto:my@email.com}{\nolinkurl{my@email.com}}
  }


  \abstract{Enter abstract here. Each new line herein must be indented, like this
line.}
  \keywords{keywords \\

    \indent Word count: X
  }





\usepackage{amsthm}
\newtheorem{theorem}{Theorem}[section]
\newtheorem{lemma}{Lemma}[section]
\theoremstyle{definition}
\newtheorem{definition}{Definition}[section]
\newtheorem{corollary}{Corollary}[section]
\newtheorem{proposition}{Proposition}[section]
\theoremstyle{definition}
\newtheorem{example}{Example}[section]
\theoremstyle{definition}
\newtheorem{exercise}{Exercise}[section]
\theoremstyle{remark}
\newtheorem*{remark}{Remark}
\newtheorem*{solution}{Solution}
\begin{document}

\maketitle

\setcounter{secnumdepth}{0}



TO BE PASTED FROM GOOGLE DOC

\section{Experiment 1}\label{experiment-1}

In our first experiment, our goal was to replicate and update the
characterization of the adult SK color system given by the World Color
Survey. We were further interested in the use of Spanish terms as
language contact and multilingualism have increased in the years since
the original World Color Survey work.

\subsection{Methods}\label{methods}

\subsubsection{Participants}\label{participants}

We recruited 39 adult participants (7 men). Most of participants were
from SK communities of the Middle Ucayali region (from Yarinacocha, San
Francisco, and Nueva Betania), but some of them were from communities of
the Lower (Paoyhan) and Upper (Puerto Belen) Ucayali. In Yarinacocha (a
small town located in the vicinity of Pucallpa), participants were
recruited in Bena Jema, a neighborhood where most of the inhabitants are
SK. All the other places where participants were recruited were native
community villages exclusively inhabited by SK people. Overall, the
sample included both somewhat urbanized SK (Yarinacocha and San
Francisco) as well as SK individuals who were still used to more
traditional activities and regular contact with the surrounding
rainforest (Nueva Betania, Paoyhan, and Puerto Belen).

The median self-reported age for participants was 38 years with a range
between 20 to 64 years of age (SD = 13.60yo). Regarding occupations,
41\% of the women were homemakers (33\% overall) and another 41\% were
artisans (33\%).
\texttt{tools::toTitleCase(as.character(as.english(filter(study1\_occupations,\ género\ ==\ \textquotesingle{}masculino\textquotesingle{}\ \&\ ocupación\ ==\ \textquotesingle{}agricultor\textquotesingle{})\$n)))}
of the 7 men were horticulturalists (43\%, 8\% overall). Four women
(12\%, 10\% overall) and three men (43\%, 8\% overall) identified as
students.

Although all adult participants spoke Shipibo-Konibo as a first
language, all were bilingual to a substantial degree. All reported
beginning to learn Spanish before early adolescence (median = 8yo, SD =
2.90y).

\subsection{Materials}\label{materials}

We used the 330 Munsell color chips as stimuli for the study. However,
only 165 chips were used for each single participant (see below). These
chips were exactly those used to collect data for the WCS. Individual
color chips were 2 cm x 2.5 cm.

\subsection{Procedure}\label{procedure}

In order to make sure that the natural light intensity would not vary
much between participants, the experiment took place indoors, near a
window or a door. The study was conducted entirely in the SK language.

Our procedure was similar to that used in the WCS (see {\textbf{???}}).
Participants were seated in front of the experimenter and introduced to
the whole procedure and the general goal of the study. Then the primary
procedure involved presenting participants with a color chip and asking
them: \enquote{What is the color of this chip?}\footnote{The SK word for
  color that we used was the Spanish word \emph{color}. In general, the
  SK language includes some castillanisms that are well-known by all
  speakers; color is one of them.} and recording their response or
responses.

One major difference between the WCS procedure and ours is that, in the
WCS, the experimenter was expected to brief participants so that they
would only provide basic color terms during the task (e.g.,
\enquote{blue} as opposed to \enquote{navy blue} or \enquote{sky-like}).
We found it rather difficult to help participants understand in a few
sentences what a basic color term was, however.\footnote{Indeed, as
  Berlin \& Kay (2009: 587-589) acknowledge, there is no straightforward
  necessary and sufficient criteria for the \enquote{basicness} of a
  color term.} Thus, we opted to let participants provide any term they
wished. If they did not provide a basic color term, we would ask further
questions to elicit a basic color term. For example, if, when presented
with a red color chip, the participant provided the term
\enquote{blood-like} (a non-basic color term), the experimenter would
ask: \enquote{Do you know of any other word to refer to the color of
this chip?} If the participant subsequently responded \enquote{dark red}
(another non-basic color term), the experimenter would further ask:
\enquote{How would you refer to this color with only one word?}
Eventually, the participant would say \enquote{red} (a basic color
term).

For some chips, participants would provide a basic color term at once;
but for others, they would first provide one or two non-basic terms
before actually providing a basic term. When participants did not
provide a basic color term after three trials (i.e., two follow-up
questions), no further questions was asked, and the experimenter
proceeded to the next chip. This method was more effortful and
time-consuming than the WCS procedure, but it improved the fluency and
the intuitiveness of the task for participants.

A second difference between our procedure and that of the WCS concerned
the number of chips each participant was presented with. In the WCS,
every participant was expected to provide color terms for each of the
330 chips of the set. As we were afraid that doing so would take too
long and that participants would find the task tedious, we decided that
the set of chips would be split in two (even and uneven numbers) and
that every participant would be randomly ascribed to one of the two
subsets. As a result, each participant was presented with only 165
chips.

\subsection{Results and Discussion}\label{results-and-discussion}

Broadly speaking, our results were quite similar to the WCS findings.
Figure 1 shows a comparison between our data (Panel A) and the WCS
(panel B). The basic level colors in our data were quite similar, as
well. All participants described at least 1 chip with the following set
of color terms: light/white (\enquote{joxo}), dark/black
(\enquote{wiso}), yellow (\enquote{panshin}), red (\enquote{joshin}),
and green/blue (\enquote{yankon}). Most (79\%) participants also used
described at least 1 chip as faded or \enquote{manxan}, referring to a
chip's saturation. In terms of overall popularity, participants on
average described 32\% of chips as \enquote{yankon} (\emph{SD} = 10\%)
followed by \enquote{joshin} (\emph{M} = 12\%, \emph{SD} = 6\%),
\enquote{joxo} (10\%, 5\%), \enquote{panshin} (9\%, 4\%),
\enquote{manxan} (7\%, 7\%), and \enquote{wiso} (6\%, 4\%).

One departure from the Berlin-Kay data was that 59\% of adults described
at least 1 chip using a Spanish-language color term, accounting for 4\%
of all responses (Figure 1, Panel C). In particular, Spanish use reached
as high as 55\% when participants were asked to label chips that English
speakers would consider to be orange. However, there was a high amount
of variability in Spanish use between subjects (\emph{M} = 4\%,
\emph{SD} = 12\%) with some subjects never responding in Spanish. One
responded in Spanish for 0\% of all trials despite all sessions being
conducted entirely in the Shipibo-Konibo language. While we can only
speculate as to this participant's motivations, it seems likely that
they were more familiar with the Spanish vocabulary or viewed it as more
precise.

Participants on average described 69\% of chips using a SK-language
basic color term like \enquote{yankon} (\emph{SD} = 22\%). Some
participants described chips using SK-language ad hoc color terms, such
as \enquote{nai} or \emph{sky} for blue chips (\emph{M} = 11\%,
\emph{SD} = 12\%), or ad hoc terms referring to saturation or luminosity
of a chip, such as \enquote{manxan} (\emph{M} = 7\%, \emph{SD} = 7\%).
Virtually all instances where a participant responded in Spanish
involved a Spanish basic color term such as \enquote{rojo} (\emph{M} =
4\%, \emph{SD} = 10\%). In other words, participants typically only
responded in Spanish to label chips into basic categories; they relied
on Shipibo-Konibo for other descriptors.

Given these data, we next moved on to exploring the development of SK
color vocabulary in childhood. Experiment 2 tests production and
comprehension of SK color terms using SK-prototypical color chips;
Experiment 3 tests children in Spanish using Spanish-prototypical chips.

\section{Experiment 2}\label{experiment-2}

In Experiment 2, we tested children on their production and
comprehension skills with a set of chips representing the prototypical
colors for common SK color terms.

\subsection{Methods}\label{methods-1}

\subsubsection{Participants}\label{participants-1}

The Pontificia Universidad Católica del Perú's Institutional Review
Board approved our study protocol. We recruited 57 5- to 11-year-old
children (23 boys). Table 1 shows the distribution of ages and genders.
Fifteen children were recruited from neighborhoods in Yarinacocha, in
the Pucallpa region of Peru, as well as in 42 children from Bawanisho, a
native community settled along the Ucayali River, south of Pucallpa.
Children were recruited either through their parents or through local
schools. When recruited at school, consent for participation was
collected from both the teachers and the parents; otherwise, only
consent from the parents was collected.

\subsubsection{Materials}\label{materials-1}

Based on findings of Experiment 1, we selected out 8 color chips that
were prototypical instances of prominent SK color terms. These color
chips were blue (WCS n°1), green (WCS n°234), red (WCS n°245), white
(WCS n°274), yellow (WCS n°297), black (WCS n°312), greeny-yellow (WCS
n°320), and purple (WCS n°325). These color chips were exactly the same
as those used in Experiment 1; the only difference was that while 330
chips were used in Experiment 1, only 8 were used in Experiment 2.

\subsubsection{Procedure}\label{procedure-1}

The production and comprehension tasks were both conducted in SK. In
both tasks, children were seated in front of the experimenter. A table
on which the color chips were display stood between them. The production
task was always performed before the comprehension task.

\textbf{Production task.} The procedure was very similar to that of
Experiment 1. Children were first introduced to the whole procedure and
the general goal of the study. It was specified that they would be
expected to provide color terms in SK (and not in Spanish). Children
were then asked: \enquote{What is the color of this chip?}. As with
adults, we used follow-up questions to elicit basic color terms when the
terms children initially provided were not basic. When children provided
Spanish color terms, the experimenter would write down their response
but further ask: \enquote{What is the name of this color in SK?} When
children replied \enquote{I don't know} to this prompt, the experimenter
would not ask further questions and would move forward to the next color
chip. As a result, responses of some children include only non-basic SK
color terms or Spanish color terms. In total, we collected production
data for 8 color chips. For each chip, the data include either one
response (when children provided a SK basic color term in the first
trial) or two or three responses (when children's initial responses were
either non-basic and/or in Spanish).

\textbf{Comprehension task.} The 8 color chips of the production task
were simultaneously displayed in front of the children. The experimenter
would then ask: \enquote{Can you give me the {[}color{]} chip?} In
total, the comprehension of 9 SK color terms was tested. The choice of
these terms was based on the findings of Experiment 1. Not all of them
were basic, but all of them stood out as being prominent in the SK color
system. The 9 terms used as prompts included: yankon (\enquote{grue}),
joshin (\enquote{red}), panshin (\enquote{yellow}), joxo
(\enquote{white), wiso (}black\enquote{), nai (}blue\enquote{), and
barin poi (}greeny-yellow\enquote{). In addition, as Experiment 1
revealed that two non-basic terms are widely used to refer to green and
purple, two words were used to test comprehension of each of these two
colors: pei/xo (}green\enquote{) and ami/pua (}purple``).

Further, Experiment 1 showed that for some of these color terms, only
one response was accurate, while for others, several responses were
equally correct. For example, only one chip could be picked up as an
instance of wiso. By contrast, four chips could be considered to be
instances of yankon (blue, green, greeny-yellow, and, to a lesser
extent, purple); two chips for joshin (red, and, to a lesser extent,
purple); and two as well for pei/xo (green, and, to a lesser extent,
green-yellow). Accuracy was coded based on the results derived from
Experiment 1: if at least 15\% of participants in Experiment 1
associated a chip with a particular label, we considered a trial to be
correct if the child made the same pairing.

When the experimenter asked children to pick up a color that was
instantiated by several chips, we followed the following procedure. The
experimenter would ask: \enquote{Can you give me the {[}color{]} chip?}
Children would then pick up a chip. The response would be registered and
the chip be taken out of the table. As a result, only 7 chips would be
remaining on the table. The experimenter would subsequently ask:
\enquote{Can you give me another {[}color{]} chip?}. Children would then
pick up a new chip. The response would be registered and the chip be
taken out of the table. The experimenter would then ask the same
question again until a total of as many times as there were correct
instances. Thus, for example, for yankon four chips would be elicited,
while for joshin, two chips would be elicited.

\subsection{Results and Discussion}\label{results-and-discussion-1}

\textbf{Production.} Children's production accuracy increased
substantially across nearly all color chips in the age range that we
tested. Figure 2, left panels shows the accuracy of children's first
production, both in SK (solid line) and in either language (dashed
line). To quantify these developmental trends, we fit two generalized
linear mixed effects models, one for the accuracy of SK production and
one for the accuracy of production in either language. Both of these
predicted accuracy as a function of the child's age, and included random
intercepts for color chip and for participant, as well as a random slope
of age by color chip. Age was a significant predictor in both models:
\(\beta = 0.42\), SE = 0.16, \(p = 0.01\) and \(\beta = 0.87\), SE =
0.17, \(p < .0001\).

Over a quarter (28\%) of all responses were given in Spanish, and the
distribution of Spanish responses was non-random. Children tended to
respond in Spanish when presented with a chip with low naming consensus
among adult participants in Experiment. We computed the naming entropy
for each chip by computing the probabilities for each chip \(c\) to be
named with a particular label \(l\) (\(p(l \mid c)\)) and then taking
\(H(c) = - \sum{p(l\mid c) \log[p(l \mid c)]}\) (see inset entropy
values by chip in Figure 2).

To assess the hypothesis that naming entropy in adults was related to
Spanish use in children, we fit a mixed effects model predicting Spanish
responses as a function of age, entropy of the chip's naming
distribution for adults, and their interaction. We included random
intercepts for color chip and for participant, but our model did not
converge with a random slope term and so we pruned this term following
our lab's standard operating procedure. We found a reliable effect of
entropy (\(\beta = -6.09\), SE = 2.38, \(p = 0.01\)) and an interaction
between age and entropy (\(\beta = -3.97\), SE = 1.49, \(p = 0.01\)),
suggesting that adults' uncertainty regarding naming was related to
children's likelihood of producing Spanish labels.

Another possible reason to use Spanish would be if children fail to
recall the proper SK color term but do know the proper mapping in the
Spanish. They may also choose to respond with a same-language but
adjacent color term )such as \enquote{joshin} for a
\emph{panshin}-colored chip). In our next analysis, we aggregate across
chips and examine the pattern of responses, categorizing them as
same-language, adjacent, and different-language.

Using a mixed-effects model, we found a significant improvement in
accuracy scores when we allowed different-language but corresponding
responses (\emph{p} \textless{} 0.001) but no significant change when
allowing for same-language but adjacent responses (\emph{p} = 0.454).

\textbf{Comprehension.}

\section{Experiment 3}\label{experiment-3}

Noting the level of bilingualism in Experiment 2, we designed Experient
3 as its complement. In Experiment 3, we tested children entirely in
Spanish with a set of chips representing prototypical colors for the
Spanish color system.

\subsubsection{Participants}\label{participants-2}

Our protocol received ethical approval from Pontificia Universidad
Católica del Perú's Institutional Review Board. Children were recruited
in a SK neighborhood of Yarinacocha (Bena Jema) as well as in Bawanisho.
As before, children were recruited either through their parents or
through the local school. When recruited at school, consent for
participation was collected from both the teachers and the parents;
otherwise, only consent from the parents was collected. Data were
collected from a total of 46 children (16 boys) who were between the
ages of 5 and 11 years old.

\subsubsection{Materials}\label{materials-2}

Even though participants of Experiment 1 were instructed to give color
terms in SK, some Spanish color terms were provided (this was especially
true of young adult participants). Based on these data and on previous
studies of Spanish color systems, we singled out 11 color chips that
were prototypical instances of prominent Peruvian Spanish color terms.
These color chips were grey (WCS n°46), pink (WCS n°65), orange (WCS
n°121), green (WCS n°234), red (WCS n°245), brown (WCS n°266), white
(WCS n°274), blue (WCS n°291), yellow (WCS n°297), black (WCS n°312) and
purple (WCS n°325). (To visualize the hue of these chips, see Appendix
1.) These color chips were exactly the same as those used in Experiment
1; the only difference was that while 330 chips were used in Experiment
1, only 11 of them were used in Experiment 3. It is worth noting that
six chips were shared between Experiment 2 and Experiment 3.

\subsubsection{Procedure}\label{procedure-2}

Since SK children are not very fluent in Spanish, the production and
comprehension tasks were both conducted in SK, and Spanish was only used
for color terms (i.e., Spanish color terms were embedded in SK
sentences). In both tasks, children were seating in front of the
experimenter; a table (on which the color chips were displayed) was
standing between them. As in Experiment 2, the production task was
always performed before the comprehension task.

Production task. The procedure was the same as that of Experiment 2.
Children were first introduced to the whole procedure and the general
goal of the study. It was specified that they would be expected to
provide color terms in Spanish (and not in SK). Children were then
asked: \enquote{what is the color of this chip?}. When children provided
SK color terms, the experimenter would write down their response but
further ask: \enquote{what is the name of this color in Spanish?}. When
children were replying \enquote{I don't know} to this prompt, the
experimenter would not ask further questions and would move forward to
the next color chip. As a result, responses of some children include
only non-basic Spanish color terms or SK color terms. In total, we
collected production data for 11 color chips. For each chip, the data
include either one response (when children provided a Spanish basic
color term in the first trial) or two or three responses (when
children's initial responses were either non-basic and/or in SK).

Comprehension task. The procedure was similar to that of the
comprehension task of Experiment 2. The 11 color chips of the production
task were simultaneously displayed in front of the children. The
experimenter would then ask: \enquote{Can you give me the \_\_\_ chip?}
(where ``\_\_\_" stands for a color term). In total, the comprehension
of 11 Spanish color terms was tested.

The choice of these terms was based on previous studies examining
Spanish color terms as well as on Experiment 1 (as we have seen, SK
adults sometimes resorted to Spanish color terms to name the color
chips). The 11 terms used as prompts included: blanco (\enquote{white}),
verde (\enquote{green}), rojo (\enquote{red}), amarillo
(\enquote{yellow}), azul (\enquote{blue}), negro (\enquote{black}),
naranja (\enquote{orange}), gris (\enquote{grey}), morado
(\enquote{purple}), marrón (\enquote{brown}), and rosa (\enquote{pink}).
Since each color term was instantiated by only one color chip, no term
required the special procedure that was followed in Experiment 2 for
yankon, joshin and pei/xo.

\subsection{Results and Discussion.}\label{results-and-discussion.}

Similar to Experiment 2, over a quarter of all responses (\emph{M} =
28\%, \emph{SD} = 18\%) were given in another language (Shipibo in this
case). There was significant variation in language-switching with some
children completing the entire task in Spanish while others responded to
upwards of 59\% of trials in Shipibo. Similar to Experiment 2, there was
no significant correlation between age and label accuracy (\emph{p} =
0.063) or between age and language-switching (\emph{p} = 0.908). Still,
we found that participants tended to respond in Shipibo when presented
with items that had low entropy among SK adults during Experiment 1 (p =
0.006). This suggests that participants across Studies 2 and 3 preferred
to respond in Shipibo when presented with a high-consensus chip and in
Spanish when shown a low-consensus chip.

\textbf{Overextensions.}

Similar to Experiment 2, we adopted alternative scoring to accommodate
language-switching from Spanish to Shipibo-Konibo and same-language
adjacent responses. Using a mixed-effects model, we did not find that
age explained a significant amount of the variation seen in accuracy
(\emph{p} = 0.124), in concordance with earlier analyses. However, we
did find that participants made use of both mapping strategies of either
providing different-language but corresponding responses (\emph{p}
\textless{} 0.001) or same-language but adjacent responses (\emph{p} =
0.002). Between Studies 2 and 3, we find frequent use of language
switching but only Experiment 3 shows significant use of same-language
but adjacent terms as well. This discrepancy, along with the lack of an
age correlation, can be due to foreign language exposure. Children may
be exposed to Spanish at a young age but do not receive any formal
Spanish education until later in adolescence. With a limited knowledge
of Spanish color terms, children may spontaneously provide Spanish color
terms during the Shipibo-language Experiment 2 but may struggle to
succeed during Spanish-language Experiment 3. This suggests that
children may rely on either strategy to communicate a color label to the
best of their knowledge set.

\textbf{Comparisons between Studies 2 \& 3. - Unfinished plots}

\section{General Discussion}\label{general-discussion}

TO BE PASTED FROM GOOGLE DOC

\newpage

\section{References}\label{references}

\begingroup
\setlength{\parindent}{-0.5in} \setlength{\leftskip}{0.5in}

\hypertarget{refs}{}

\endgroup






\end{document}
