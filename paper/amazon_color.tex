\documentclass[,man,floatsintext]{apa6}
\usepackage{lmodern}
\usepackage{amssymb,amsmath}
\usepackage{ifxetex,ifluatex}
\usepackage{fixltx2e} % provides \textsubscript
\ifnum 0\ifxetex 1\fi\ifluatex 1\fi=0 % if pdftex
  \usepackage[T1]{fontenc}
  \usepackage[utf8]{inputenc}
\else % if luatex or xelatex
  \ifxetex
    \usepackage{mathspec}
  \else
    \usepackage{fontspec}
  \fi
  \defaultfontfeatures{Ligatures=TeX,Scale=MatchLowercase}
\fi
% use upquote if available, for straight quotes in verbatim environments
\IfFileExists{upquote.sty}{\usepackage{upquote}}{}
% use microtype if available
\IfFileExists{microtype.sty}{%
\usepackage{microtype}
\UseMicrotypeSet[protrusion]{basicmath} % disable protrusion for tt fonts
}{}
\usepackage{hyperref}
\hypersetup{unicode=true,
            pdftitle={The Development of Color Terms in Shipibo-Konibo Children},
            pdfauthor={Danielle Kellier, Martin Fortier, Maria Fernández Flecha, \& Michael C. Frank},
            pdfkeywords={keywords},
            pdfborder={0 0 0},
            breaklinks=true}
\urlstyle{same}  % don't use monospace font for urls
\usepackage{graphicx,grffile}
\makeatletter
\def\maxwidth{\ifdim\Gin@nat@width>\linewidth\linewidth\else\Gin@nat@width\fi}
\def\maxheight{\ifdim\Gin@nat@height>\textheight\textheight\else\Gin@nat@height\fi}
\makeatother
% Scale images if necessary, so that they will not overflow the page
% margins by default, and it is still possible to overwrite the defaults
% using explicit options in \includegraphics[width, height, ...]{}
\setkeys{Gin}{width=\maxwidth,height=\maxheight,keepaspectratio}
\IfFileExists{parskip.sty}{%
\usepackage{parskip}
}{% else
\setlength{\parindent}{0pt}
\setlength{\parskip}{6pt plus 2pt minus 1pt}
}
\setlength{\emergencystretch}{3em}  % prevent overfull lines
\providecommand{\tightlist}{%
  \setlength{\itemsep}{0pt}\setlength{\parskip}{0pt}}
\setcounter{secnumdepth}{0}
% Redefines (sub)paragraphs to behave more like sections
\ifx\paragraph\undefined\else
\let\oldparagraph\paragraph
\renewcommand{\paragraph}[1]{\oldparagraph{#1}\mbox{}}
\fi
\ifx\subparagraph\undefined\else
\let\oldsubparagraph\subparagraph
\renewcommand{\subparagraph}[1]{\oldsubparagraph{#1}\mbox{}}
\fi

%%% Use protect on footnotes to avoid problems with footnotes in titles
\let\rmarkdownfootnote\footnote%
\def\footnote{\protect\rmarkdownfootnote}


  \title{The Development of Color Terms in Shipibo-Konibo Children}
    \author{Danielle Kellier\emph{\textsuperscript{1}, Martin
Fortier}\textsuperscript{2}, Maria Fernández Flecha\textsuperscript{3},
\& Michael C. Frank\textsuperscript{4}}
    \date{}
  
\shorttitle{Color Terms in Shipibo-Konibo Children}
\affiliation{
\vspace{0.5cm}
\textsuperscript{1} University of Pennsylvania\\\textsuperscript{2} PSL Research University\\\textsuperscript{3} Pontificia Universidad Católica del Perú\\\textsuperscript{4} Stanford University}
\keywords{keywords\newline\indent Word count: X}
\usepackage{csquotes}
\usepackage{upgreek}
\captionsetup{font=singlespacing,justification=justified}

\usepackage{longtable}
\usepackage{lscape}
\usepackage{multirow}
\usepackage{tabularx}
\usepackage[flushleft]{threeparttable}
\usepackage{threeparttablex}

\newenvironment{lltable}{\begin{landscape}\begin{center}\begin{ThreePartTable}}{\end{ThreePartTable}\end{center}\end{landscape}}

\makeatletter
\newcommand\LastLTentrywidth{1em}
\newlength\longtablewidth
\setlength{\longtablewidth}{1in}
\newcommand{\getlongtablewidth}{\begingroup \ifcsname LT@\roman{LT@tables}\endcsname \global\longtablewidth=0pt \renewcommand{\LT@entry}[2]{\global\advance\longtablewidth by ##2\relax\gdef\LastLTentrywidth{##2}}\@nameuse{LT@\roman{LT@tables}} \fi \endgroup}


\usepackage{lineno}

\linenumbers

\authornote{* these authors contributed equally.

MCF was supported by a Jacobs Foundation Fellowship. We are especially
grateful to Paul Kay for assistance throughout the study.

Correspondence concerning this article should be addressed to Martin
Fortier*, Postal address. E-mail:
\href{mailto:martin.fortier@ens.fr}{\nolinkurl{martin.fortier@ens.fr}}}

\abstract{
Enter abstract here. Each new line herein must be indented, like this
line.


}

\usepackage{amsthm}
\newtheorem{theorem}{Theorem}[section]
\newtheorem{lemma}{Lemma}[section]
\theoremstyle{definition}
\newtheorem{definition}{Definition}[section]
\newtheorem{corollary}{Corollary}[section]
\newtheorem{proposition}{Proposition}[section]
\theoremstyle{definition}
\newtheorem{example}{Example}[section]
\theoremstyle{definition}
\newtheorem{exercise}{Exercise}[section]
\theoremstyle{remark}
\newtheorem*{remark}{Remark}
\newtheorem*{solution}{Solution}
\begin{document}
\maketitle

TO BE PASTED FROM GOOGLE DOC

\section{Experiment 1}\label{experiment-1}

In our first experiment, our goal was to replicate and update the
characterization of the adult SK color system given by the World Color
Survey. We were further interested in the use of Spanish terms as
language contact and multilingualism have increased in the years since
the original World Color Survey work.

\subsection{Methods}\label{methods}

\subsubsection{Participants}\label{participants}

We recruited 39 adult participants (7 men). Most of participants were
from SK communities of the Middle Ucayali region (from Yarinacocha, San
Francisco, and Nueva Betania), but some of them were from communities of
the Lower (Paoyhan) and Upper (Puerto Belen) Ucayali. In Yarinacocha (a
small town located in the vicinity of Pucallpa), participants were
recruited in Bena Jema, a neighborhood where most of the inhabitants are
SK. All the other places where participants were recruited were native
community villages exclusively inhabited by SK people. Overall, the
sample included both somewhat urbanized SK (Yarinacocha and San
Francisco) as well as SK individuals who were still used to more
traditional activities and regular contact with the surrounding
rainforest (Nueva Betania, Paoyhan, and Puerto Belen).

The median self-reported age for participants was 38 years with a range
between 20 to 64 years of age (SD = 13.60yo). Regarding occupations,
41\% of the women were homemakers (33\% overall) and another 41\% were
artisans (33\%). Three of the 7 men were horticulturalists (43\%, 8\%
overall). Four women (12\%, 10\% overall) and three men (43\%, 8\%
overall) identified as students.

Although all adult participants spoke Shipibo-Konibo as a first
language, all were bilingual to a substantial degree. All reported an
introduction to the Spanish language before early adolescence (\emph{M}
= 7.80yo, \emph{SD} = 2.90y). Participant age and reported age of
introduction to Spanish were positively correlated; younger participants
reported learning Spanish at an early age although all participants
reported introductions before early adolescence (\emph{r} = 0.43,
\emph{p} = 0.01).

\subsection{Materials}\label{materials}

We used the 330 Munsell color chips as stimuli for the study. However,
only 165 chips were used for each single participant (see below). These
chips were exactly those used to collect data for the WCS. Individual
color chips were 2 cm x 2.5 cm.

\subsection{Procedure}\label{procedure}

In order to make sure that the natural light intensity would not vary
much between participants, the experiment took place indoors, near a
window or a door. The study was conducted entirely in the SK language.

Our procedure was similar to that used in the WCS (see Kay, Berlin,
Maffin, Merrifield, \& Cook, 2009, pp. 585--591). Participants were
seated in front of the experimenter and introduced to the whole
procedure and the general goal of the study. Then the primary procedure
involved presenting participants with a color chip and asking them:
\enquote{What is the color of this chip?}\footnote{The SK word for color
  that we used was the Spanish word \emph{color}. In general, the SK
  language includes some castillanisms that are well-known by all
  speakers; color is one of them.} and recording their response or
responses.

One major difference between the WCS procedure and ours is that, in the
WCS, the experimenter was expected to brief participants so that they
would only provide basic color terms during the task (e.g.,
\enquote{blue} as opposed to \enquote{navy blue} or \enquote{sky-like}).
However, we found it rather difficult to help participants understand in
a few sentences what a basic color term was.\footnote{Indeed, as Berlin
  \& Kay (2009: 587-589) acknowledge, there is no straightforward
  necessary and sufficient criteria for the \enquote{basicness} of a
  color term.} Thus, we opted to let participants provide any term they
wished. If they did not provide a basic color term, we would ask further
questions to elicit a basic color term. For example, if the participant
provided the term \enquote{blood-like} (a non-basic color term) when
presented with a red color chip, the experimenter would ask: \enquote{Do
you know of any other word to refer to the color of this chip?} If the
participant subsequently responded \enquote{dark red} (another non-basic
color term), the experimenter would further ask: \enquote{How would you
refer to this color with only one word?} Eventually, the participant
would say \enquote{red} (a basic color term).

For some chips, participants would provide a basic color term at once;
but for others, they would first provide one or two non-basic terms
before actually providing a basic term. When participants did not
provide a basic color term after three trials (i.e., two follow-up
questions), no further questions was asked, and the experimenter
proceeded to the next chip. This method was more effortful and
time-consuming than the WCS procedure, but it improved the fluency and
the intuitiveness of the task for participants.

A second difference between our procedure and that of the WCS concerned
the number of chips each participant was presented with. In the WCS,
every participant was expected to provide color terms for each of the
330 chips of the set. As we were afraid that doing so would take too
long and that participants would find the task tedious, we decided that
the set of chips would be split in two (even and uneven numbers) and
that every participant would be randomly ascribed to one of the two
subsets. As a result, each participant was presented with only 165
chips.

\subsection{Results and Discussion}\label{results-and-discussion}

\begin{figure}
\centering
\includegraphics{amazon_color_files/figure-latex/adultfigure-1.pdf}
\caption{\label{fig:adultfigure}(A and B) Plots of the modal term given for
a particular chip. Color coordinates were represented in 2-D Munsell
space. Modal responses were given by SK adults during (A) the original
World Color Survey and during (B) our Experiment 1. (C) Heat map of
prevalence of Spanish-language responses during Experiment 1. Legends
for all three subplots located in the bottom-right quadrant.}
\end{figure}

Broadly speaking, our results were quite similar to the WCS findings.
Figure 1 shows a comparison between our data (Panel A) and the WCS
(panel B). The basic level colors in our data were quite similar, as
well. All participants described at least 1 chip with the following set
of color terms: light/white (\enquote{joxo}), dark/black
(\enquote{wiso}), yellow (\enquote{panshin}), red (\enquote{joshin}),
and green/blue (\enquote{yankon}). Most (79\%) participants also used
described at least 1 chip as faded or \enquote{manxan}, referring to a
chip's saturation. In terms of overall popularity, participants on
average described 32\% of chips as \enquote{yankon} (\emph{SD} = 10\%)
followed by \enquote{joshin} (\emph{M} = 12\%, \emph{SD} = 6\%),
\enquote{joxo} (10\%, 5\%), \enquote{panshin} (9\%, 4\%),
\enquote{manxan} (7\%, 7\%), and \enquote{wiso} (6\%, 4\%).

One departure from the Berlin-Kay data was that 59\% of adults described
at least 1 chip using a Spanish-language color term, accounting for 4\%
of all responses (Figure 1, Panel C). In particular, Spanish use reached
as high as 55\% when participants were asked to label chips that English
speakers would consider to be orange. However, there was a high amount
of variability in Spanish use between subjects (\emph{M} = 4\%,
\emph{SD} = 12\%) which neither participant age (\emph{p} = 0.87) nor
reported age of Spanish introduction (\emph{p} = 0.56) failed to
predict. Some subjects never responding in Spanish whereas one
participant responded in Spanish for 71\% of all trials despite all
sessions being conducted entirely in the Shipibo-Konibo language. While
we can only speculate as to this participant's motivations, it seems
likely that they were more familiar with Spanish color vocabulary or
viewed Spanish color terms as more precise descriptors.

\hypertarget{htmlwidget-eaef49d4a484bc22a8b4}{}

--\textgreater{}

Participants on average described 69\% of chips using a SK basic color
term like \enquote{yankon} (\emph{SD} = 22\%). Some participants
described chips using SK ad-hoc color terms, such as \enquote{nai} or
\emph{sky} for blue chips (\emph{M} = 11\%, \emph{SD} = 12\%), or ad hoc
terms referring to saturation or luminosity of a chip, such as
\enquote{manxan} (\emph{M} = 7\%, \emph{SD} = 7\%). Virtually all
instances where a participant responded in Spanish involved a Spanish
basic color term such as \enquote{rojo} (\emph{M} = 4\%, \emph{SD} =
10\%). In other words, participants typically only responded in Spanish
to label chips into basic categories; they relied on Shipibo-Konibo for
other descriptors.

Given these data, we moved on to exploring the development of SK color
vocabulary in childhood. Experiment 2 tests production and comprehension
of SK color terms using SK-prototypical color chips; Experiment 3 tests
children in Spanish using Spanish-prototypical chips.

\section{Experiment 2}\label{experiment-2}

In Experiment 2, we tested children on their production and
comprehension skills with a set of chips representing the prototypical
colors for common SK color terms.

\subsection{Methods}\label{methods-1}

\subsubsection{Participants}\label{participants-1}

\begin{table}[tbp]
\begin{center}
\begin{threeparttable}
\caption{\label{tab:unnamed-chunk-2}Demographics of participants in Experiment 2.}
\begin{tabular}{lll}
\toprule
Age Group & \multicolumn{1}{c}{N} & \multicolumn{1}{c}{Male}\\
\midrule
5 & 3 (5\%) & 1 (33\%)\\
6 & 8 (14\%) & 3 (38\%)\\
7 & 12 (21\%) & 4 (33\%)\\
8 & 15 (26\%) & 5 (33\%)\\
9 & 10 (18\%) & 5 (50\%)\\
10 & 4 (7\%) & 2 (50\%)\\
11 & 5 (9\%) & 3 (60\%)\\
\bottomrule
\end{tabular}
\end{threeparttable}
\end{center}
\end{table}

The Pontificia Universidad Católica del Perú's Institutional Review
Board approved our study protocol. We recruited 57 5- to 11-year-old
children (23 boys). Table 1 shows the distribution of ages and genders.
Fifteen children were recruited from neighborhoods in Yarinacocha, in
the Pucallpa region of Peru, as well as in 42 children from Bawanisho, a
native community settled along the Ucayali River, south of Pucallpa.
Children were recruited either through their parents or through local
schools. When recruited at school, consent for participation was
collected from both the teachers and the parents; otherwise, only
consent from the parents was collected.

\subsubsection{Materials}\label{materials-1}

Based on findings of Experiment 1, we selected out 8 color chips that
were prototypical instances of prominent SK color terms. These color
chips were blue (WCS n°1), green (WCS n°234), red (WCS n°245), white
(WCS n°274), yellow (WCS n°297), black (WCS n°312), greeny-yellow (WCS
n°320), and purple (WCS n°325). These color chips were exactly the same
as those used in Experiment 1; the only difference was that adult
participants in Study 1 were presented with these chips along the rest
of their assigned 165 chip set. Child participants only had these 8
chips.

\subsubsection{Procedure}\label{procedure-1}

The production and comprehension tasks were both conducted in SK. In
both tasks, children were seated in front of the experimenter. A table
on which the color chips were display stood between them. The production
task was always performed before the comprehension task.

\textbf{Production task.} The procedure was very similar to that of
Experiment 1. Children were first introduced to the whole procedure and
the general goal of the study. It was specified that they would be
expected to provide color terms in SK (and not in Spanish). Children
were then asked: \enquote{What is the color of this chip?}. As with
adults, we used follow-up questions to elicit basic color terms when the
terms children initially provided were not basic. When children provided
Spanish color terms, the experimenter would write down their response
but further ask: \enquote{What is the name of this color in SK?} When
children replied \enquote{I don't know} to this prompt, the experimenter
would not ask further questions and would move forward to the next color
chip. As a result, responses of some children include only non-basic SK
color terms or Spanish color terms. In total, we collected production
data for 8 color chips. For each chip, the data include either one
response (when children provided a SK basic color term in the first
trial) or two or three responses (when children's initial responses were
either non-basic and/or in Spanish).

Further, Experiment 1 showed that for some of these color terms, only
one response was accurate, while for others, several responses were
equally correct. For example, responses during Experiment 1 to a
particular purple chip ranged from red to blue with some using the terms
ami (\enquote{flower}) or pua (\enquote{yam}) as common descriptors.
Accuracy was coded based on the results derived from Experiment 1: if at
least 15\% of participants in Experiment 1 labeled a chip with a
particular term, we considered a trial to be correct if the child made
the same pairing, regardless of whether the term as a basic or ad-hoc
color term.

\textbf{Comprehension task.} The 8 color chips of the production task
were simultaneously displayed in front of the children. The experimenter
would then ask: \enquote{Can you give me the {[}color{]} chip?} In
total, the comprehension of 9 SK color terms was tested. The choice of
these terms was based on the findings of Experiment 1. Not all of them
were basic, but all of them stood out as being prominent in the SK color
system. The 9 terms used as prompts included: yankon
(\enquote{green/blue}), joshin (\enquote{red}), panshin
(\enquote{yellow}), joxo (\enquote{white), wiso (}black\enquote{), nai
(}blue\enquote{), and barin poi (}greeny-yellow\enquote{). In addition,
as Experiment 1 revealed that two non-basic terms are widely used to
refer to green and purple, two words were used to test comprehension of
each of these two colors: pei/xo (}green\enquote{) and ami/pua
(}purple``).

When the experimenter asked children to pick up a color that was
instantiated by several chips, we followed the following procedure. The
experimenter would ask: \enquote{Can you give me the {[}color{]} chip?}
Children would then pick up a chip. The response would be registered and
the chip be taken out of the table. As a result, only 7 chips would be
remaining on the table. The experimenter would subsequently ask:
\enquote{Can you give me another {[}color{]} chip?}. Children would then
pick up a new chip. The response would be registered and the chip be
taken out of the table. The experimenter would then ask the same
question again until a total of as many times as there were correct
instances. Thus, for example, for yankon four chips would be elicited,
while for joshin, two chips would be elicited. Like the preceding
production task, accuracy was scored based on responses given in Study
1. If a child chose a particular chip, their choice was deemed accurate
if at least 15\% of participants during Study 1 made the same chip-label
pairing.

\subsection{Results and Discussion}\label{results-and-discussion-1}

\includegraphics{amazon_color_files/figure-latex/childfigure-1.pdf}
\includegraphics{amazon_color_files/figure-latex/childfigure-2.pdf}

\subsubsection{Production}\label{production}

Children's production accuracy increased substantially across nearly all
color chips in the age range that we tested. Figure 2, top panel shows
the accuracy of children's first production, both in SK (solid line) and
in either language (dashed line). To quantify these developmental
trends, we fit two generalized linear mixed effects models, one for the
accuracy of SK production and one for the accuracy of production in
either language. Both of these predicted accuracy as a function of the
child's age, and included random intercepts for color chip and for
participant, as well as a random slope of age by color chip. Age was a
significant predictor in both models: \(\beta = 1.05\), SE = 0.28,
\(p = 0\) and \(\beta = 1.11\), SE = 0.23, \(p < .0001\).

Over a quarter (28\%) of all responses were given in Spanish, and the
distribution of Spanish responses was non-random. Children tended to
respond in Spanish when presented with a chip with low naming consensus
among adult participants in Experiment. As an exploratory analysis, we
attempted to quantify low naming consensus using naming entropy
(following {\textbf{???}}). We computed the naming entropy for each chip
by computing the probabilities for each chip \(c\) to be named with a
particular label \(l\) (\(p(l \mid c)\)) and then taking
\(H(c) = - \sum{p(l\mid c) \log[p(l \mid c)]}\) (see inset entropy
values by chip in Figure 2).

To assess the hypothesis that naming entropy in adults was related to
Spanish use in children, we fit a mixed effects model predicting Spanish
responses as a function of age, entropy of the chip's naming
distribution for adults, and their interaction. We included random
intercepts for color chip and for participant, but our model did not
converge with a random slope term and so we pruned this term following
our lab's standard operating procedure. We found a reliable effect of
entropy (\(\beta = -6.09\), SE = 2.38, \(p = 0.01\)) and an interaction
between age and entropy (\(\beta = -3.97\), SE = 1.49, \(p = 0.01\)),
suggesting that adults' uncertainty regarding naming was related to
children's likelihood of producing Spanish labels.

One reason to use Spanish would be if children fail to recall the proper
SK color term but do know the proper mapping in the Spanish. But another
possibilitiy is that children may have more imprecise representations
and choose to respond with a same-language but adjacent color term (such
as \enquote{joshin} for a \emph{panshin}-colored chip). In our next
analysis, following ({\textbf{???}}), we aggregate across color chips
and examine the pattern of children's first responses, categorizing them
as same-language, adjacent, and different-language. This analysis is
shown in Figure 3, left panel.

We fit a mixed-effects model predicting correct performance with
predictors specified as above, but including only random intercepts for
participants due to convergence issues). We found a significant
improvement in accuracy scores when we allowed different-language but
corresponding responses (\emph{p} \textless{} 0.001) but no significant
change when allowing for same-language but adjacent responses (\emph{p}
= 0.454). This result suggests that children's incorrect responding was
not due to imprecise knowledge of SK terms.

\begin{figure}
\centering
\includegraphics{amazon_color_files/figure-latex/study23accuracyplots-1.pdf}
\caption{\label{fig:study23accuracyplots}Proportion of accurate responses
when applying different accuracy criteria, by age and experiment. Points
show the mean for a 2-year age group (chosen arbitrarily for
visualization) with 95\% confidence intervals. Lines show a loess
smoothing function.}
\end{figure}

\subsubsection{Comprehension}\label{comprehension}

Children's accuracy in the comprehension task increased substantially
across nearly all color chips in the age range that we tested. Figure 3,
top panel shows the accuracy of children's first production, both in for
strict accuracy (solid line) and including chips for adjacent colors
(dashed line). To quantify these developmental trends, we fit two
generalized linear mixed effects models, one for the accuracy of SK
production and one for choosing the accurate or adjacent chips. Both of
these predicted accuracy as a function of the child's age, and included
random intercepts for color chip and for participant, as well as a
random slope of age by color chip. Age was a significant predictor in
both models: \(\beta = 0.60\), SE = 0.18, \(p = 0.00\) and
\(\beta = 0.67\), SE = 0.19, \(p < .0001\).

\section{Experiment 3}\label{experiment-3}

Noting the level of bilingualism in the SK population, we designed
Experient 3 as its complement. Due to the length of these experiments,
however, as well as the task demands involved in testing the same
children sequentially in both languages, we chose to perform this next
experiment with a separate group of children. In Experiment 3, we tested
children entirely in Spanish with a set of chips representing
prototypical colors for the Spanish color system.

\subsubsection{Participants}\label{participants-2}

\begin{table}[tbp]
\begin{center}
\begin{threeparttable}
\caption{\label{tab:unnamed-chunk-5}Demographics of participants in Experiment 3.}
\begin{tabular}{lll}
\toprule
Age Group & \multicolumn{1}{c}{N} & \multicolumn{1}{c}{Male}\\
\midrule
5-years-old & 2 (4\%) & 1 (50\%)\\
6-years-old & 2 (4\%) & 0 (0\%)\\
7-years-old & 11 (24\%) & 4 (36\%)\\
8-years-old & 9 (20\%) & 1 (11\%)\\
9-years-old & 11 (24\%) & 4 (36\%)\\
10-years-old & 8 (17\%) & 3 (38\%)\\
11-years-old & 3 (7\%) & 3 (100\%)\\
\bottomrule
\end{tabular}
\end{threeparttable}
\end{center}
\end{table}

As with Study 2, our protocol received ethical approval from Pontificia
Universidad Católica del Perú's Institutional Review Board. Children
were recruited in a SK neighborhood of Yarinacocha (Bena Jema) as well
as in Bawanisho. As before, children were recruited either through their
parents or through the local school. When recruited at school, consent
for participation was collected from both the teachers and the parents;
otherwise, only consent from the parents was collected. Data were
collected from a total of 46 children (16 boys) between the ages of 5
and 11 years old.

\subsubsection{Materials}\label{materials-2}

Even though participants in Experiment 3 were instructed to give color
terms in SK, some Spanish color terms were provided (this was especially
true of younger adult participants, who were more proficient in
Spanish). Based on these data and on previous studies of Spanish color
systems, we singled out 11 color chips that were prototypical instances
of prominent Peruvian Spanish color terms. These color chips were grey
(WCS n°46), pink (WCS n°65), orange (WCS n°121), green (WCS n°234), red
(WCS n°245), brown (WCS n°266), white (WCS n°274), blue (WCS n°291),
yellow (WCS n°297), black (WCS n°312) and purple (WCS n°325). These
color chips were exactly the same as those used in Experiment 1; the
only difference was that while 330 chips were used in Experiment 1, only
11 of them were used in Experiment 3. Six chips were shared between
Experiment 2 and Experiment 3.

\subsubsection{Procedure}\label{procedure-2}

Since SK children are not very fluent in Spanish, the production and
comprehension tasks were both conducted in SK, and Spanish was only used
for color terms (i.e., Spanish color terms were embedded in SK
sentences). As in Experiment 2, the production task was always performed
before the comprehension task.

\textbf{Production task.} The procedure was the same as that of
Experiment 2. Children were first introduced to the whole procedure and
the general goal of the study. It was specified that they would be
expected to provide color terms in Spanish (and not in SK). Children
were then asked: \enquote{what is the color of this chip?} When children
provided SK color terms, the experimenter would write down their
response but further ask: \enquote{what is the name of this color in
Spanish?} When children replied \enquote{I don't know} to this prompt,
the experimenter would not ask further questions and would move forward
to the next color chip. As a result, responses by some children include
only non-basic Spanish color terms or SK color terms. For each chip, the
data include either one response (when children provided a Spanish basic
color term in the first trial) or two or three responses (when
children's initial responses were either non-basic and/or in SK).

\textbf{Comprehension task.} The procedure was identical to that of the
comprehension task of Experiment 2, with the exception of the set of
chips and labels. In total, the comprehension of 11 Spanish color terms
was tested. The choice of these terms was based on previous studies
examining Spanish color terms as well as on Experiment 1. The 11 terms
used as prompts included: blanco (\enquote{white}), verde
(\enquote{green}), rojo (\enquote{red}), amarillo (\enquote{yellow}),
azul (\enquote{blue}), negro (\enquote{black}), naranja
(\enquote{orange}), gris (\enquote{grey}), morado (\enquote{purple}),
marrón (\enquote{brown}), and rosa (\enquote{pink}). Since each color
term was instantiated by only one color chip, no term required the
special procedure that was followed in Experiment 2 for the ambiguous
terms.

\subsection{Results and Discussion}\label{results-and-discussion-2}

\subsubsection{Production}\label{production-1}

The results of the production task are shown in Figure 2, bottom panel.
Qualitatively, we saw smaller developmental effets. As in Experiment 2,
we fit two generalized linear mixed effects models, one for the accuracy
of SK production and one for the accuracy of production in either
language. Both of these predicted accuracy as a function of the child's
age, and included random intercepts for color chip and for participant,
as well as a random slope of age by color chip. Age was not a
significant predictor in either model: \(\beta = 0.32\), SE = 0.20,
\(p = 0.11\) and \(\beta = 0.43\), SE = 0.16, \(p < .0001\).

Similar to Experiment 2, over a quarter of all responses (\emph{M} =
28\%, \emph{SD} = 18\%) were given in another language (Shipibo in this
case). There was significant variation in language-switching with some
children completing the entire task in Spanish while others responded to
upwards of 59\% of trials in Shipibo.In addition, similar to Experiment
2, we found that participants tended to respond in Shipibo when
presented with items that had low entropy among SK adults during
Experiment 1 (\(p\) = 0.006). This suggests that participants across
Studies 2 and 3 preferred to respond in Shipibo when presented with a
high-consensus chip and in Spanish when shown a low-consensus chip.

Also following our analysis in Experiment 2, we adopted alternative
scoring to accommodate language-switching from Spanish to Shipibo-Konibo
and same-language adjacent responses. Results are shown in Figure 3,
right panel. Using a mixed-effects model, we did not find that age
explained a significant amount of the variation seen in accuracy
(\emph{p} = 0.124), in concordance with earlier analyses. However, we
did find that participants made use of \emph{both} alternative
strategies, either providing SK responses (\emph{p} \textless{} 0.001)
or same-language, adjacent responses (\emph{p} = 0.002). In other words,
in both Experiment 2 and 3, we find frequent use of language switching
but only Experiment 3 shows significant use of djacent terms as well.

We speculate that the findings of Experiment 3 -- the lack of
developmental increases and the increasing use of adjacent Spanish terms
-- are a function of the nature of second-language exposure in Spanish.
SK children are often exposed to Spanish at a young age, but they do not
receive any formal Spanish education until later in adolescence. With a
limited knowledge of Spanish color terms, children may spontaneously
provide Spanish color terms during the SK-language Experiment 2 for
those mappings they know but may still struggle to succeed during
Spanish-language Experiment 3. More generally, we see children relying
on a mixture of strategies to communicate colors even in the absence of
complete knowledge in either language.

\subsection{Comprehension}\label{comprehension-1}

Unlike the production task for Experiment 3, children's accuracy in the
comprehension task increased substantially across nearly all color chips
in the age range that we tested. Figure 3, top panel shows the accuracy
of children's first production, both in for strict accuracy (solid line)
and including chips for adjacent colors (dashed line). To quantify these
developmental trends, we fit two generalized linear mixed effects
models, one for the accuracy of SK production and one for choosing the
accurate or adjacent chips. Both of these predicted accuracy as a
function of the child's age, and included random intercepts for color
chip and for participant, as well as a random slope of age by color
chip. Age was a significant predictor in both models: \(\beta = 0.64\),
SE = 0.22, \(p = 0.00\) and \(\beta = 0.49\), SE = 0.17, \(p < .0001\).

\section{General Discussion}\label{general-discussion}

TO BE PASTED FROM GOOGLE DOC

\newpage

\section{References}\label{references}

\begingroup
\setlength{\parindent}{-0.5in} \setlength{\leftskip}{0.5in}

\hypertarget{refs}{}
\hypertarget{ref-berlin2009}{}
Kay, P., Berlin, B., Maffin, L., Merrifield, W. R., \& Cook, R. (2009).
\emph{The world color survey}. Stanford, CA: Center for the Study of
Language; Information.

\endgroup


\end{document}
